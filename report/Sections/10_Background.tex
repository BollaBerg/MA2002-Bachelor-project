\section{Introduction}
The climate is changing at an ever-increasing pace. With 19 of the 20 hottest years on record happening since the turn of the millennium \cite{NASA-climate}, it is adamantly clear that something must be done. One proposed part of the solution for climate change is carbon capture and storage, where CO2 is captured -- whether from the air or before it is released -- and stored for long periods of time \cite{climate-mitigation}. One way of storing CO2 safely for longer periods of time is by utilizing subsurface, geological formations, such as depleted oil- and gas reservoirs. In order for this to actually help climate change mitigation over time, the leakage rate must less than 1\% per 1000 years \cite{Shaffer-CO2-capture}. To ensure this can be achieved, accurate and precise simulation tools are required.

The first step of any simulation workflow starts with an accurate model of the domain. Subsurface geological formations are typically highly heterogeneous, often consisting of many layers with different physical properties \cite{UPR_thesis}. This is further complicated by large-scale geological features such as faults. To store CO2 in the reservoirs, there must also be a way to create wells leading into the reservoirs, adding an additional feature our models must handle. We therefore need a modelling system that can handle heterogeneity and represent several different, often crossing, physical features, while still being computationally efficient enough to be usable.

Several methods for computer modelling have been attempted. The earliest methods consisted of Cartesian grids, extended from slices of the domain. Later methods included corner-point grids \cite{corner-point} -- Cartesian grids with irregular polygons, which further evolved into unstructured grids, with PEBI grids being a popular choice \cite{UPR_thesis}. In unstructured grids, wells are typically represented as cell centroids, while faults are traced by faces in the grid.

There exists several tools for simulating geological systems. One of these tools is the MATLAB Reservoir Simulation Toolbox (MRST). MRST includes a module for constructing PEBI grids -- UPR -- which supports faults through face alignment of the cells, and wells through cell centroid alignment \cite{UPR_chapter}. Due to how UPR creates its background grid, the software is only capable of constructing small- to medium-sized grids, and may struggle in certain cases. The primary goal of this thesis has been to develop an extension to the UPR module, by replacing the grid refinement algorithm with mesh creation in Gmsh, thus enabling users of MRST to easily create more complex and detailed grids. We create three methods for automatically creating meshes in Gmsh, then create a MATLAB module interfacing all three methods. Special care is taken to ensure the module follows these principles, and create conforming grids.

The thesis is structured as follows. \autoref{sec:Theory} goes through some necessary background theory, particularily focused on Voronoi diagrams and PEBI grids. \autoref{sec:Software} discuss existing software. The section introduce both Gmsh and MRST, and includes both limitations and examples of the two software packages. It also discusses UPR further. In \autoref{sec:combining}, I first discuss the existing method for combining Gmsh and UPR. I then introduce a new package for combining Gmsh and MRST in \autoref{sec:own_software}, including a guide of installation and use, as well as an in-depth discussion of the implementation details of the new package. It naturally includes examples of how the package can be used, as well as a number of plots and illustrations. The thesis then concludes by discussing how the package can be improved going forward in \autoref{sec:Future}.