\section{Introduction}
The climate is changing at an ever-increasing pace. With 19 of the 20 hottest year on record happening since the turn of the millennium \cite{NASA-climate}, it is adamantly clear that something must be done. One proposed part of the solution for climate change is carbon capture and storage, where CO2 is captured -- whether from the air or before it is released -- and stored for long periods of time \cite{climate-mitigation}. One way of storing CO2 safely for longer periods of time is by utilizing subsurface, geological formations, such as existing oil- and gas reservoirs. In order for this to actually help climate change mitigation over time, the leakage rate must less than 1\% per 1000 years \cite{Shaffer-CO2-capture}. To achieve this, accurate and precise modelling is required.

The first step of any modelling timeline starts with an accurate representation of the domain. Subsurface reservoirs are typically nonhomogenous, often consisting of many layers with different physical properties \cite{UPR_thesis}. This is further complicated by large-scale geological features such as faults. In order to store CO2 in the reservoirs, there must also be a way to create wells leading into the reservoirs, adding an additional feature our models must handle. In order to create a computer representation of this domain, we need a system that can handle inhomogeneity and represent several different, often crossing, physical features, while still being computationally efficient enough to be usable.

% Several methods have been attempted ...
% PEBI Grids
% MRST

The thesis is structured as follows. \autoref{sec:Theory} goes through some necessary background theory, particularily focused on Voronoi diagrams and PEBI grids. \autoref{sec:Software} discuss existing software. The section introduce both Gmsh and MRST, and includes both limitations and examples of the two software packages. It also discusses UPR further. In \autoref{sec:combining}, I first discuss the existing method for combining Gmsh and UPR. I then introduce a new package for combining the Gmsh and MRST in \autoref{sec:own_software}, including a guide of installation and use, as well as an in-depth discussion of the implementation details of the new package. It naturally includes examples of how the package can be used, as well as a liberal amount of plots and illustrations. The thesis then rounds up by discussing how the package can be improved going forward, in \autoref{sec:Future}.