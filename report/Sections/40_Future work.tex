\section{Future work}
\label{sec:Future}
With as large user base and feature set as Gmsh has, there is significant potential for exciting features to be used in \verb|gmsh4mrst|. Whether through advanced mesh refinement, improved triangulation algorithms, or any other features of Gmsh, the potential of expanding the use of Gmsh is definitely present.

One key requirement for future improvement of \verb|gmsh4mrst| is by gaining a better understanding of the requirements, wants and needs of potential users. As my domain knowledge is limited, it is hard to fully understand what features are and are not useful, so the first step in any future development is simply to gain users and collect feedback. This way, any developers can ensure that \verb|gmsh4mrst| is optimized for the right work, while any unused features or unneeded parameters may be removed.

One interesting topic to consider is using Gmsh to create structured background grids, instead of the unstructured triangulations \verb|gmsh4mrst| currently produced. This could be achieved using a combination of embeddings and transfinite grids, but would likely require significant manual work in order to make robust.

When looking forward, one clear development of \verb|gmsh4mrst| is to expand the grid generation to 3D. This is something both Gmsh and MRST can handle, and while it would require at least some adaptations of the code, it could potentially open up a new dimension of opportunities when it comes to combining the two tools. As modelling targets usually are three-dimensional, this would also expand the usability of the package significantly.

Finally, MATLAB provides functionality for calling code written in C++, and Gmsh provides a C++ API. Although most of the functionality for Gmsh is already written in C++ and only called from Python, a porting of \verb|gmsh4mrst| to C++ could potentially lead to increased speed. While this is not needed for models on a smaller scale, it could be beneficial for larger, complex domains with numerous faults and wells. As \verb|gmsh4mrst| provides a MATLAB package separate from the Python package, a port of the Gmsh-calling code could be done without changing anything in how the package is used from MATLAB.

One goal of \verb|gmsh4mrst| was to create an open-source project that can be continuously developed as the needs of the software develop. The software is currently distributed with the same license as Gmsh, which allows anyone to use, change and distribute the software as they want, as long as future distributions are also open-source. With this in mind, I hope this project can be used as a starting point for future development, and that \verb|gmsh4mrst| will become a useful tool for researchers using MRST.
