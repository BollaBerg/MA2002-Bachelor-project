\section{Abstract}
Creating an accurate, discrete grid representation of a real-life, continuous domain can be challenging. This is especially true when trying to model geological formations, as they tend to be highly heterogeneous, often with crossing faults or wells. Accurate grids are, however, crucial for creating realistic simulations. This drives a need for software that can quickly and efficiently generate grids conforming to constraints of the real-life domain.

In this thesis, I discuss theory behind existing methods for creating grids, including a discussion of triangulations and their relationship to grids. I then present two existing tools for creating grids -- Gmsh and MRST, and how they can be used. Finally, I introduce a new software module for combining the two -- \verb|gmsh4mrst|, and discuss how it can be used, as well as how it can be developed further in the future.

This thesis was written in collaboration with SINTEF Digital.

\section{Oppsummering}
Det kan være svært utfordrende å skape presise, nøyaktige grid-representasjoner av virkelige, kontinuerlige domener. Dette er spesielt sant når det kommer til modellering av geologiske formasjoner, ettersom de har en tendens til å være svært heterogene, ofte med kryssende forkastninger og brønner. Nøyaktige representasjoner er likevel kritiske for å skape realistiske simuleringer. Dette skaper et behov for programvare som kjapt og effektivt kan skape grids som representerer det virkelige domenet.

I denne oppgaven diskuterer jeg teorien bak eksisterende metoder for å skape grids, inkludert en diskusjon om trianguleringer og deres relasjon til grids. Jeg presenterer videre to eksisterende verktøy for å skape grids -- Gmsh og MRST, og hvordan de kan brukes per dags dato. Avslutningsvis introduserer jeg et nytt program for å kombinere disse to -- \verb|gmsh4mrst|, og diskuterer hvordan dette kan brukes, samt hvordan det kan videreutvikles i fremtiden.

Denne oppgaven ble skrevet i samarbeid med SINTEF Digital.